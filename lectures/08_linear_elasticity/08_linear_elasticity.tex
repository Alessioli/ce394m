%**************************************************************************************
% License:
% CC BY-NC-SA 4.0 (http://creativecommons.org/licenses/by-nc-sa/4.0/)
%**************************************************************************************

\documentclass[handout]{beamer}

\mode<presentation> {

\usetheme{Madrid}

% Burnt orange
\definecolor{burntorange}{rgb}{0.8, 0.33, 0.0}
\colorlet{beamer@blendedblue}{burntorange}
% Pale yellow
\definecolor{paleyellow}{rgb}{1.0, 1.0, 0.953}
\setbeamercolor{background canvas}{bg=paleyellow}
% Secondary and tertiary palett
\setbeamercolor*{palette secondary}{use=structure,fg=white,bg=burntorange!80!black}
\setbeamercolor*{palette tertiary}{use=structure,fg=white,bg=burntorange!60!black}

% To remove the footer line in all slides uncomment this line
%\setbeamertemplate{footline}
% To replace the footer line in all slides with a simple slide count uncomment this line
%\setbeamertemplate{footline}[page number]

% To remove the navigation symbols from the bottom of all slides uncomment this line
%\setbeamertemplate{navigation symbols}{}
}

\usepackage{amsmath}
\usepackage{bm}
\usepackage{breqn}
\usepackage{graphicx} % for figures
\usepackage{subcaption} % for subplots 
\usepackage[labelsep=space,tableposition=top]{caption}
\renewcommand{\figurename}{Fig.} 
\usepackage{cleveref}
\usepackage{caption,subcaption}% http://ctan.org/pkg/{caption,subcaption}
\usepackage{booktabs} % Allows the use of \toprule, \midrule and \bottomrule in tables
\usepackage{multirow}

% To print 2 slides on a page
%\usepackage{handoutWithNotes}
%\pgfpagesuselayout{2 on 1}[border shrink=2mm]
%----------------------------------------------------------------------------------------
%	TITLE PAGE
%----------------------------------------------------------------------------------------
% The short title appears at the bottom of every slide, the full title is only on the title page
\title[CE394M: Linear Elasticity]{CE394M: Linear Elasticity} 
\author{Krishna Kumar} % name
\institute[UT Austin] % institution 
{
University of Texas at Austin \\
\medskip
\textit{
  \url{krishnak@utexas.edu}} % Your email address
}
\date{\today} % Date, can be changed to a custom date

\begin{document}

\begin{frame}
\titlepage % title page as the first slide
\end{frame}

\begin{frame}
 % Table of contents slide, comment this block out to remove it
 \frametitle{Overview}
  %Throughout your presentation, if you choose to use \section{} and \subsection{} 
  %commands, these %will automatically be printed on this slide as an overview 
 \tableofcontents
\end{frame}

%----------------------------------------------------------------------------------------
% slides
%----------------------------------------------------------------------------------------
\section{Review of vector calculus}
%----------------------------------------------------------------------------------------
\begin{frame}
\frametitle{Isotropic linear elastic stress-strain relations}
The linear relationship between the stress and strain tensor is a linear one. The stress component
is a linear combination of the strain tensor:
\begin{equation*}
	\begin{split}
		\sigma_{ij} = C_{ij11}\varepsilon_{11} + C_{ij12}\varepsilon_{12} + C_{ij13}\varepsilon_{13} + \\
			C_{ij21}\varepsilon_{21} + C_{ij22}\varepsilon_{22} + C_{ij23}\varepsilon_{23} + \\ C_{ij31}\varepsilon_{31} + C_{ij32}\varepsilon_{32} + C_{ij33}\varepsilon_{33}
	\end{split}
\end{equation*} 
The most general form for \textit{linear} stress-strain relations for a \textit{Cauchy elastic}
material is given by:
\mode<beamer>{
	\begin{equation*}
		\sigma_{ij} = B_{ij} + C_{ijkl} \varepsilon_{kl}
	\end{equation*}
}
\mode<handout>{
	\vspace{1cm}
} 
	Where $B_{ij}$ is the components of initial stress tensor corresponding to the initial strain free (when all strain components $\varepsilon_{kl} = 0$). $C_{ijkl}$ is the tensor of material \textit{elastic constants}.
	
	If it is assumed that the initial strain free state corresponds to an \textit{initial stress free state}, that is $B_{ij} = 0$, the equations reduces to:
\mode<beamer>{	
	\begin{equation*}
		\sigma_{ij} = C_{ijkl} \varepsilon_{kl}
	\end{equation*}
}  
\mode<handout>{
	\vspace{1.5cm}
} 
\end{frame}
\end{document}