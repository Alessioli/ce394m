\documentclass[a4paper,12pt]{article}
\usepackage{graphicx}
\usepackage[left=30mm, right=30mm, top=30mm, bottom=35mm]{geometry}
\usepackage{amsmath}
\usepackage{siunitx}
\usepackage{fancyhdr}
\usepackage{url}
\pagestyle{fancy}
%-------------------------------------------------------------------------------
\lhead{\textbf{Spring 2019}}
\rhead{\textbf{CE394M Advanced Analysis in Geotechnical Engineering}}
\cfoot{\thepage}
%-------------------------------------------------------------------------------

\begin{document}
\begin{centering}
	\textbf{
		Project 2: Implemntation of Modified Cam Clay in Python\\
		Assigned: 24th April 2019\\
		Due: 10th May 2019\\
	}
\end{centering}

\vspace{1em}

Explain in detail the formulation of the Modified Cam-Clay model and derive the D-matrix
of the model. Simulate two undrained triaxial tests using the following Cam-Clay model
parameters:
\begin{align*}
M &= 0.95\\
N &= 2.7\\
\lambda &= 0.16\\
\kappa &= 0.06\\
\nu &= 0.2
\end{align*}

\begin{enumerate}
	\item Isotopically consolidated undrained compression test of normally consolidated clay.
	Normally consolidated to 100 kPa and then sheared in undrained conditions. For the undrained shear part of the test:
	\begin{enumerate}
		\item Determine the initial void ratio before shearing.
		\item Plot deviatoric stress $q$ versus axial strain $\varepsilon_a$
		\item Plot the stress path in $q-p^\prime$ plane and the state path in $e - \ln p^\prime$ plane
		\item Plot excess pore pressure $u$ versus axial strain
	\end{enumerate}

	\item Isotopically consolidated undrained compression test of overconsolidated clay.
	Normally consolidated to 450 kPa isotopically, unloaded isotropically to 100 kPa
	and then sheared in undrained conditions. For the undrained shear part of the test:
	\begin{enumerate}
		\item Determine the initial void ratio before shearing.
		\item Plot deviatoric stress $q$ versus axial strain $\varepsilon_a$
		\item Plot the stress path in $q-p^\prime$ plane and the state path in $e - \ln p^\prime$ plane
		\item Plot excess pore pressure $u$ versus axial strain
	\end{enumerate}
	Note:
	\begin{enumerate}
		\item 	Using zero volumetic strain condition (undrained), $\varepsilon_r = - \varepsilon_a /2$ 
		\item Compute $\sigma_1^\prime, \sigma_3^\prime (=\sigma_2^\prime)$ using the D-matrix and then compute $p^\prime$ and $q$
		\item For the overconsolidated condition, use the elasto-plastic D matrix after the stress
		state reaches the yield surface. Use the elastic D matrix before yielding.
	\end{enumerate}
	\item Discuss how to simulate an isotropically consolidated `drained' triaxial compression test.
		
\end{enumerate}

Elasto-plastic relation:
\begin{equation*}
d\sigma^\prime = \left[D^e - \frac{D^e\left(\frac{\partial G}{\partial \sigma^\prime}\right)\left(\frac{\partial F}{\partial \sigma^\prime}\right)^T D^e}{-\left(\frac{\partial F}{\partial W_p}\right)\left(\frac{\partial W_p}{ \partial \varepsilon^p}\right)^T\left(\frac{\partial G}{\partial \sigma^\prime}\right) + \left(\frac{\partial F}{\partial \sigma^\prime}\right)^T D^e \left(\frac{\partial G}{\partial \sigma^\prime}\right)}\right] d\varepsilon
\end{equation*}

Yield function of modified Cam-Clay:
\begin{equation*}
F = \frac{q^2}{M^2} - p^\prime p_c + p^2 = 0
\end{equation*}

\begin{equation*}
\frac{\partial F}{\partial \sigma^\prime} = \frac{\partial F}{\partial p^\prime}\frac{\partial p^\prime}{\partial \sigma^\prime} + \frac{\partial F}{\partial q}\frac{\partial q}{\partial \sigma^\prime}
\end{equation*}

\begin{equation*}
\frac{\partial F}{\partial p^\prime} = 2 p - p_c
\end{equation*}

\begin{equation*}
\frac{\partial F}{\partial q} = 2q / M^2
\end{equation*}

\begin{equation*}
\frac{\partial F}{\partial \sigma^\prime} = \begin{bmatrix}
1/3 \\
1/3 \\
1/3 \\
0 \\
0 \\
0 \\
\end{bmatrix}
\end{equation*}

\begin{equation*}
\frac{\partial q}{\partial \sigma^\prime} = (3/2q) \begin{bmatrix}
\sigma_xx - p \\
\sigma_yy - p\\
\sigma_zz - p\\
2 \sigma_{xy} \\
2 \sigma_{yz} \\
2 \sigma_{zx} \\
\end{bmatrix}
\end{equation*}

Calculate $(\partial F / \partial p_c)(dp_c/d\varepsilon_v^p)(\partial F/\partial p)$ by differentiating the relevant terms.
\begin{equation*}
\frac{\partial p_c}{\partial d\varepsilon_v^p} = \frac{vp_c}{(\lambda - \kappa)}
\end{equation*}
\end{document}

