\documentclass[a4paper,12pt]{article}
\usepackage{graphicx}
\usepackage[left=30mm, right=30mm, top=30mm, bottom=35mm]{geometry}
\usepackage{amsmath}
\usepackage{siunitx}
\usepackage{fancyhdr}
\usepackage{url}
\pagestyle{fancy}
%-------------------------------------------------------------------------------
\lhead{\textbf{Spring 2019}}
\rhead{\textbf{CE394M Advanced Analysis in Geotechnical Engineering}}
\cfoot{\thepage}
%-------------------------------------------------------------------------------

\begin{document}
\begin{centering}
	\textbf{
		Assignment 5: PLAXIS FEA of circular footing\\
		Assigned: 4th March 2019\\
		Due: 15th March 2019\\
	}
\end{centering}

\vspace{1em}

You may obtain access to Plaxis through a Virtual Desktop.  Information about gaining access can be found at \url{ http://caee.utexas.edu/students/itss/43-students/it/386-virtualdesktops}.  

 
\begin{enumerate}

	\item 	Familiarize yourself with the documentation of the program PLAXIS, with particular emphasis on the PLAXIS “2D 2019 – Tutorial Manual” and PLAXIS “2D 2019 – Material Models Manual”.  These can be found at \url{https://www.plaxis.com/support/manuals/plaxis-2d-manuals/}.  Please read and complete the following the Tutorial on the settlement of a Circular Footing on Sand (Sections 1.1-1.3).
	
	``Play'' with the program to ensure that you can set up, solve, and interpret a problem through the pre- and post-processor of PLAXIS.  
	
	Write a short report including the following:
	
	\begin{enumerate}
		\item Described the elements used, average mesh size, soil properties and the solver.
		
		\item Final deformed meshes for the rigid and flexible footings and the load versus displacement plot for the center of the flexible footing.

		\item Explain how the results were validated.
	\end{enumerate}
	
\end{enumerate}

\end{document}

